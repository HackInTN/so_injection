\section{Petits rappels de compilation}

\frame{\tableofcontents[currentsection]}

\begin{frame}{Le processus de compilation}
	\begin{itemize}
		\item Analyses (lexicale, syntaxique, sémantique)
		\item Génération de  code intermédiaire, optimisations...
		\item Génération des fichiers objets
		\item Edition des liens
	\end{itemize}
\end{frame}

\begin{frame}{Edition des liens}
	\begin{itemize}
		\item Objectif : créer un éxécutable à partir de fichiers objets (.o,.obj,.a)
		\item Les fichiers objets contiennent du code et la table de leurs symboles
		\item Trois cas:
		\begin{itemize}
		
		\item Edition de liens entre plusieurs objets internes au programme \\
		    ex: A.o $\rightarrow$ B.o $\rightarrow$ C.a
		\item Edition statique de liens avec des objets externes (sur l'OS) \\
		    ex: A.o $\rightarrow$ /usr/lib/x86\_64-linux-gnu/libcrypt.a
		\item Edition dynamique de liens avec des objets externes (chargés à l'exécution) \\
			ex: A.o $\rightarrow$ /lib/i386-linux-gnu/libc.so.6
		\end{itemize}
	\end{itemize}
\end{frame}

\section{Les bibliothèques dynamiques}

\frame{\tableofcontents[currentsection]}

\begin{frame}{Où ?}
	\begin{itemize}
		\item Sur Linux: .so (Shared object), emplacements: /lib, /usr/lib, /usr/local/lib
		\item Windows: .dll (Dynamic link library), C:\textbackslash Windows\textbackslash system32
		\item OSX: who cares ?
	\end{itemize}
\end{frame}

\begin{frame}{Quelques commandes utiles}
	\begin{itemize}
	    \item file <executable>: permet de savoir si le programme utilise des librairies dynamiques
		\item ldd <binaire>: affiche la liste des dépendances dynamiques
		\item ltrace <executable>: permet d'intercepter les appels à des librairies dynamiques
		\item readelf -s <binaire>: affiche la liste des symboles.
		\item gcc -shared -fPIC prog.c -o prog.so  : compiler un .so sur linux/gcc
	\end{itemize}
\end{frame}
